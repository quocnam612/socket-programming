\section{Proof of Concept (PoC)}
\subsection{Mô tả}
Mục tiêu của PoC là xác minh rằng:
\begin{itemize}
    \item Ứng dụng C++ có thể gửi yêu cầu sinh văn bản tới mô hình LLM đã được cài đặt trong Ollama.
    \item Kết quả trả về có thể được đọc, xử lý và sử dụng tiếp trong chương trình.
    \item Cả hai phương thức giao tiếp (CLI và REST API) đều hoạt động ổn định, có thể mở rộng cho các ứng  lớn hơn.
\end{itemize}
Trước khi chạy chương trình phải kiểm tra xem ollama có đang chạy bằng lệnh:
\begin{lstlisting}[language=bash]
ollama serve\end{lstlisting}
Kiểm tra các mô hình đã tải bằng lệnh:
\begin{lstlisting}[language=bash]
ollama list\end{lstlisting}

\subsection{Gọi Ollama CLI}
Phương pháp này dùng command-line của Ollama:
\begin{lstlisting}[language=bash]
ollama run llama3.1 "Hello"\end{lstlisting}
Trong C++, có thể dùng lệnh std::system trong <cstdlib>
\begin{lstlisting}[language=c++]
void cli::autoAsk(const string& model) {
    string prompt;
    do {
        cout << RED_COLOR << "You: " << BLUE_COLOR;
        getline(cin, prompt);

        if (prompt == "exit") break;
        cout << GREEN_COLOR << "Ollama: \n" << BLUE_COLOR << flush;
        string cmd = "ollama run " + model + " \"" + cli::escapeQuotes(prompt) + "\"";
        system(cmd.c_str());
    }
    while (true);
}\end{lstlisting}
\subsection{Gọi Ollama REST API}
Phương pháp này giao tiếp bằng REST API:
\begin{lstlisting}[language=c++]
generate(const string& model, const string& prompt) {
    httplib::Client cli("localhost", 11434);

    json body = {
        {"model", model},
        {"prompt", prompt},
        {"stream", false}
    };

    auto res = cli.Post("/api/generate", body.dump(), "application/json");

    Response out;
    if (!res) {
        out.response = "[Error] HTTP request failed. Is Ollama running?";
        return out;
    }
    if (res->status != 200) {
        out.response = "[Error] HTTP " + to_string(res->status) + ": " + res->body;
        return out;
    }

    json resp_json = json::parse(res->body);
    out.response   = resp_json.value("response", "");
    return out;
}\end{lstlisting}

